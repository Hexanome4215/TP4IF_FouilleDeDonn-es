\subsection{\eng{Clustering}}
Une fois la phase de sélection des données, il reste évidemment a effectuer le \eng{clustering}. Cette étape consiste à classifier les données en paquet afin de pouvoir en extraire des relations entre eux.

Cette étape se réalise grâce à de nombreux algorithmes. Profitons de cette partie pour étudier les spécificité de 3 grands algorithmes : 
\begin{itemize}
	\item Le \eng{clustering} hiérarchique
	\item La méthode \eng{K-Mean}
	\item La méthode \eng{Fuzzy C-Mean}
\end{itemize}

\subsubsection{Détermination du nombre de \eng{clusters}}
Le nombre de \eng{clusters} n'est pas laissé libre aux différents algorithmes et il faut prendre une décision cohérente.

Parfois, ce nombre est fixé : \eg lorsque le client commande une classification en 3 groupes \og maigre\fg, \og normal\fg, \og obèse\fg.

Cependant, ce n'est pas toujours (et même rarement) le cas. Il existe cependant une méthode pour déterminer de manière formelle un nombre de cluster adapté au jeu de données.

\paragraph{\eng{Clustering} hiérarchique}
Encore une fois, l'utilisation du \eng{clustering} hiérarchique est recommandé pour prendre des décisions.
\todo{SINGLE : Bras}
\todo{COMPLETE et AVERAGE : Paquets, et mieux adapté à \eng{K-Means}}
\todo{Distance : Utilisation des gaps plus que de la dérivé.}

\subsection{\eng{Clustering}}

\subsubsection{\eng{Hierarchical Clustering}}

\subsubsection{\eng{K-Mean}}

\subsubsection{\eng{Fuzzy C-Mean}}

\subsubsection{Arbre de décision}

\paragraph{Normalisation}
\todo{Elle est parfois inutile (dans le cas de valeur en pourcentages \eg}
