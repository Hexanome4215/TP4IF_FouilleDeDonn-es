\section{Description des données}
Le jeu de données provient du site \url{http://worldbank.org}. Il concerne 210 pays et fournis des informations comme :
\begin{itemize}
	\item Le pourcentage du \gls{RNB} généré par l'agriculture, l'industrie ou les services.
	\item La quantité de produits importés et exportés.
	\item Des informations démographiques et géographiques comme la densité et l'age de la population ou la surface du pays.
	\item Des informations sanitaires comme la quantité de personnes infectées par le \gls{VIH}.
	\item Des centaines d'autres indicateurs.
\end{itemize}
Par ailleurs, ce jeu de donnée semble être un très bon entrainement pour nous exercer au \eng{datamining}. En effet, en plus d'exhiber des données très concrètes pour nous, il est évident que des corrélations sont présentes. Enfin, il pourra être très intéressant de voir ces corrélations d'abord à l'échelle du monde, puis à l'échelle de l'Europe par exemple. 

