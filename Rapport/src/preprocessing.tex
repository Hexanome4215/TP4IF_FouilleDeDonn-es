\subsection{Filtrage des données}
Contrairement à ce que nous avions cru avant de commencer, il faut très souvent élaguer son \eng{dataset} afin de donner plus de sens aux valeurs. Étudions les différentes phases et techniques associées à ce pré-traitement.

\subsubsection{Réduction des dimensions}
Une des problématique majeure du \eng{datamining} s'appelle la \emph{malédiction de la dimensionalité}. Cette \og malédiction\fg~ provient du fait que plus le nombre de dimensions (\ie le nombre de colonnes) du jeu de donnée croit, moins la notions de distance\footnote{Quelque soit la fonction de distance utilisée (Manhattan, Euclidienne, Sebestyen, \etc)} n'a de sens. En effet, les distances ont tendance à se réduire proportionnellement avec le nombre de dimensions. Or, les algorithmes de \eng{clustering} utilisent cette distance comme critère principal de choix à l'appartenance à tel ou tel \eng{cluster}. Les résultats risquent donc fortement de perdre en sens et en précision.

Voyons alors comment nous pouvons, dans le cadre d'un jeu de donnée contenant beaucoup de colonne, palier à ce défi technique.

\paragraph{Choix des colonnes}
Dans un premier temps, il est fondamental de ne sélectionner que les colonnes qui nous intéressent.

Dans le cadre d'une étude de \eng{datamining} réalisée par des professionnels, l'expérience dans le domaine analysé joue beaucoup et permettra de faire un choix cohérent et judicieux des dimensions à conserver. Cependant, il est fréquent que les données ne soient pas dans le domaine d'expérience du chercheur ou pire, très abstraites. Dans ce cas, il faut faire appel à d'autre techniques.

\paragraph{Correlation}
Il peut être très intéressant d'utiliser une matrice de correlation afin de mettre en valeurs les colonnes qui pourraient nous êtres utiles, et surtout celles qui nous sont inutiles.
\image{CorrelationMatrix}{0.5}

La lecture de cette matrice est très simple. Plus les couleurs (rouge ou bleu) sont foncées, plus la corrélation (\ie la facilité à déduire les données d'une colonne à partir d'une autre) entre les colonnes est forte. 

Il est évident que les diagonales soient de la couleur la plus foncée possible car une colonne est toujours corrélée avec elle même.

Dans cet exemple, il est facile de déduire de la matrice que le sexe et le mois de mesure n'ont finalement pas beaucoup d'impacte sur les autres paramètres. 

Nous pouvons donc les ignorer dans notre \eng{clustering}.

\paragraph{PCA}
La \gls{PCA} est une méthode de réduction automatique des dimensions. Elle a l'avantage de préserver \textsl{mathématiquement} le plus d'information possible. Pour cela, elle essaie de chercher la meilleur combinaison linaire des différentes colonnes afin d'atteindre, au choix : \emph{un nombre fixé de dimension} ou \emph{une conservation de X\% de l'information}.

Bien que pratique, elle est (tout du moins à notre niveau d'expertise en \eng{datamining}) relativement dangereuse car les combinaisons linéaires qui résultent d'une \gls{PCA} perdent beaucoup de leur sémantique. Aussi, il est assez compliqué d'obtenir la formule de sortie (\ie l'application linéaire qui a été définie par la \gls{PCA}) afin de l'interpréter.

\subsubsection{Valeurs manquantes}
Certains \eng{dataset} contiennent des lignes n'exhibant pas toutes les colonnes. Il y a donc \og des cases vides\fg. Ce problème anodin peut pourtant poser des problèmes et il faut adopter une des stratégies suivantes :
\begin{itemize}
	\item Supprimer l'ensemble des lignes où au moins une valeur manque : C'est la solution la plus propre mais le jeu de donnée risque de se réduire drastiquement.
	\item Remplacer la valeur manquante : Soit par une moyenne des autres valeurs, soit par une valeur \og label\fg~ (\eg \texttt{missing}), soit même par une valeur aléatoire.
\end{itemize}
La meilleur solution reste toute de même de bien sélectionner ses colonnes, et de supprimer les lignes contenant une valeur manquante. Il faut cependant trouver un équilibre entre l'exactitude des valeurs utilisée et la taille du \gls{dataset}.

\subsubsection{\eng{Outliers}}
Un \eng{outlier} est une valeur atypique du jeu de donnée risquant de biaiser l'analyse que nous pourrions en faire. 

Cependant, elles peuvent aussi aider à traiter un problème dans un cas plus général. C'est pourquoi leur élimination (ou conversation) est problématique et fait appel aussi bien au bon sens, qu'à une parfaite compréhension des données ainsi qu'à l'expérience.

\paragraph{Recherche \og à la main\fg}
La première méthode de recherche a l'avantage d'être simple. Elle consiste simplement à effectuer une recherche visuelle des \eng{outliers} via un \eng{scatter plot} pour les jeux de données de dimension inférieur à 2 et via un \eng{parallel coordinates} sinon. L'humain étant naturellement doué fort pour détecter les valeurs extrêmes, les \eng{outliers} devraient être assez simple à exclure. 

\begin{multicols}{2}
	\image[Un exemple de diagramme \og Scatter plot\fg]{ScatterPlot}{0.4} \columnbreak
	\image[Un exemple de diagramme \og ParallelCoordinates\fg]{ParallelCoordinates}{0.4}
\end{multicols}

\paragraph{\eng{Clustering} hiérarchique}
Une méthode efficace et qui me semble un peu plus rigoureuse consiste à utiliser un \eng{clustering} hiérarchique afin d'en extraire les \eng{outliers}.

La méthode est très simple : \citation{Chaque point ou cluster est progressivement absorbé par le cluster le plus proche.} 

En effet, le dernier point ou petit groupe de point qui se fait \og coller\fg~ en dernier représente un \eng{outlier}.

Sur la figure \ref{fig4}, il est évident que le point le plus a gauche est un \eng{outlier}. En effet, il s'est collé en dernier, et ce, à une distance presque égale au rayon du cluster contenant tous les autres points.
\image[Exemple de dendogramme extrait d'un \eng{clustering} hiérarchique]{Dendrogram}{1}

\paragraph{\eng{Box plot}}
C'est encore un moyen très rapide de démasquer les \eng{outliers} car ils font figurer sur le même diagramme : la médian, les quartiles et les décile. Nous avons donc toutes les données statistiques nécessaire afin d'évaluer l'appartenance d'une point à notre futur jeu de données.
\image[Exemple de boîte à moustache]{BoxPlot}{1}

\paragraph{Méthodes statistiques}
Une des dernières méthodes (que nous n'utilisons pas sur Knime) est une approche statistique des données. Il est prouvé que de nombreux phénomènes suivent une loi normale. De plus, nous connaissons très bien cette loi et des mathématiciens ont pu modéliser l'atypique afin de fournir des méthodes automatiques d'exclusion des \eng{outliers}. 

Nous ne rentrerons pas plus dans le détail mais parmi ces méthodes, nous pouvons citer le critère de \textsc{Peirce}, celui de \textsc{Chauvenet} ou encore le test de \textsc{Grubb}.
