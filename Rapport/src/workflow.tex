\section{Méthode de travail}
\subsection{Filtrage des données}
\paragraph{Normalisation}
\todo{Elle est parfois inutile (dans le cas de valeur en pourcentages \eg}
\paragraph{Valeurs manquantes}
\paragraph{\eng{Outliers}}
\todo{Il faut les chercher en 1D, 2D, et plus (parallel coordinates}
\todo{Utilisation du clustering hiérarchique pour les identifier : Repérer les derniers collé}
\todo{Utilisation des box plots}
\paragraph{Correlation}

\subsection{Réduction des dimensions}
\paragraph{Choix des colonnes}
\paragraph{PCA}

\subsection{Détermination du nombre de \eng{clusters}}
\todo{Intro ou le nombre est connu}
\paragraph{\eng{Clustering} hiérarchique}
\todo{SINGLE : Bras}
\todo{COMPLETE et AVERAGE : Paquets, et mieux adapté à \eng{K-Means}}
\todo{Distance : Utilisation des gaps plus que de la dérivé.}

\subsection{\eng{Clustering}}
\todo{Faire la liste et les spécificité des différentes méthodes de \eng{clustering}}
\subsubsection{\eng{Hierarchical Clustering}}
\subsubsection{\eng{K-Mean}}
\subsubsection{\eng{Fuzzy C-Mean}}
