\newpage
\section{Secteurs d'activité des différents pays}
Pour cette première étude de cas, nous avons choisi d'utiliser les données afin de caractériser l'activité d'un pays en fonction de :
\todo{Ajouter le nom de la colonne correspondante}
\begin{itemize}
	\item La taille du pays
	\item La population
	\item Le \gls{PIB}
	\item Le \gls{RNB}
	\item Les exports industriels
\end{itemize}

Notre démarche dans cette analyse est simple : Après avoir établi un \eng{clustering} sur le pourcentage du \gls{PIB} dû à l'agriculture, l'industrie et les services, nous avons créer un arbre de décision prenant pour critère de sélection les données citées ci-dessus.

\subsection{Sélection des données}
\todo{Ajouter l'environnement subimages pour une meilleur intégration}
Après avoir sélectionné les colonnes qui nous intéressent ( \ie \texttt{Agriculture, value added (\% of GDP)}, \texttt{Industry, value added (\% of GDP)} et \texttt{Service, value added (\% of GDP)}), nous obtenons la matrice suivante : % \wrappedimag

Le diagramme de \todo{boite à chaussure??} est : % \wrappedimage

Enfin le diagramme \eng{parallel coordinates} : 

\todo{Guinée : 95\% de son PIB est industriel}

\subsection{\eng{Clustering}}
\subsubsection{\eng{Clustering} hiérarchique}
Le choix du type de \eng{clustering} hiérarchique est, \apriori délicat. Voici les résultats obtenus avec un \eng{clustering} hiérarchique :
\begin{description}
	\item [SINGLE] % \image
	\item [COMPLETE] % \image
\end{description}

\subsubsection{\eng{Clustering K-Means}}

\paragraph{Vérification de la stabilité du \eng{clustering}}
\todo{Inclure le tableau de vérification de l'entropie}

\subsection{Interprétation}
\subsubsection{Arbre de décision}
